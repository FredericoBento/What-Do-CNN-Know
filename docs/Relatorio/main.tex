% **************************************************************************************************************
% A Classic Thesis Style
% An Homage to The Elements of Typographic Style
%
% Copyright (C) 2012 Andr\'e Miede http://www.miede.de
%
% If you like the style then I would appreciate a postcard. My address 
% can be found in the file ClassicThesis.pdf. A collection of the 
% postcards I received so far is available online at 
% http://postcards.miede.de
%
% License:
% This program is free software; you can redistribute it and/or modify
% it under the terms of the GNU General Public License as published by
% the Free Software Foundation; either version 2 of the License, or
% (at your option) any later version.
%
% This program is distributed in the hope that it will be useful,
% but WITHOUT ANY WARRANTY; without even the implied warranty of
% MERCHANTABILITY or FITNESS FOR A PARTICULAR PURPOSE.  See the
% GNU General Public License for more details.
%
% You should have received a copy of the GNU General Public License
% along with this program; see the file COPYING.  If not, write to
% the Free Software Foundation, Inc., 59 Temple Place - Suite 330,
% Boston, MA 02111-1307, USA.
%
% **************************************************************************************************************
% Note:
%    * You must not use "u etc. in strings/commands that will be spaced out (use \"u or real umlauts instead)
%    * New enumeration (small caps): \begin{aenumerate} \end{aenumerate}
%    * For margin notes: \marginpar or \graffito{}
%    * Do not use bold fonts in this style, it is designed around them
%    * Use tables as in the examples
%    * See classicthesis-preamble.sty for useful commands
% **************************************************************************************************************
% To Do:
%   * [high] Check this out: http://www.golatex.de/koma-script-warnung-in-verbindung-mit-listings-package-t2058.html
%   * [medium] mathbb in section-titles/chapter-titles => disappears somehow in headlines!!!
% **************************************************************************************************************
\documentclass[ twoside,openright,titlepage,numbers=noenddot,headinclude,%1headlines,% letterpaper a4paper
                footinclude=true,cleardoublepage=plain,abstractoff, % <--- obsolete, remove (todo)
                BCOR=5mm,paper=a4,fontsize=11pt,%11pt,a4paper,%
                portuguese,
                dottedtoc, % adicionar pontinhos na lista de conteúdos
                ]{scrreprt}


% UTF-8 support with latin9 (ISO-8859-9) = latin1+"Euro sign"
\PassOptionsToPackage{utf8}{inputenc}   
\usepackage{inputenc}  
 
% ****************************************************************************************************
% Personal data and user ad-hoc commands
% ****************************************************************************************************
\newcommand{\myTitle}{O que é que as redes convolucionais conseguem aprender \xspace}
\newcommand{\myDegree}{Licenciatura em Eng.ª Informática\xspace}
\newcommand{\myNameOne}{Estudante Frederico Assunção de Sá Bento\xspace}
\newcommand{\myNumberOne}{2211012}
\newcommand{\myNameTwo}{Estudante Pedro Nuno Tempero Serafim\xspace}
\newcommand{\myNumberTwo}{2211084}


\newcommand{\myProfOne}{Professor Doutor Carlos Fernando de Almeida Grilo
\href{mailto:luis.conde@ipleiria.pt}{(carlos.grilo@ipleiria.pt)}\xspace}
\newcommand{\myProfTwo}{Professor Doutor José Carlos Bregieiro Ribeiro  \href{mailto:hugo.costelha@ipleiria.pt}{(jose.ribeiro@ipleiria.pt)}\xspace}
\newcommand{\myProfThree}{Professor Doutor Rolando Lúcio Germano Miragaia
\href{mailto:catarina.reis@ipleiria.pt}{(rolando.miragaia@ipleiria.pt)}\xspace}
% \newcommand{\myProfTwo}{Professor Doutor XXXX XXXXX XXXXX \href{mailto:xxxxxx@ipleiria.pt}{(xxxxxx@ipleiria.pt)}\xspace}

\newcommand{\myFaculty}{Politécnico de Leiria\xspace}
\newcommand{\mySchool}{Escola Superior de Tecnologia e Gestão\xspace}
\newcommand{\myDepartment}{Departamento de Engenharia Informática\xspace}
\newcommand{\myLocation}{Leiria\xspace}

\newcommand{\myTime}{Março de 2024\xspace}
\newcommand{\mySchoolYear}{2023 -- 2024\xspace}
\newcommand{\myVersion}{versão 0.1\xspace}            
                
                
%*******************************************************
% Note: Make all your adjustments in here
%*******************************************************
\input{config}

\usepackage{lipsum}
\usepackage{colortbl}
\usepackage{pgfplots}
\usepackage{pgfplotstable}



\pgfplotstableset{
    /color cells/min/.initial=0,
    /color cells/max/.initial=1000,
    /color cells/textcolor/.initial=,
    %
    % Usage: 'color cells={min=<value which is mapped to lowest color>, 
    %   max = <value which is mapped to largest>}
    color cells/.code={%
        \pgfqkeys{/color cells}{#1}%
        \pgfkeysalso{%
            postproc cell content/.code={%
                %
                \begingroup
                %
                % acquire the value before any number printer changed
                % it:
                \pgfkeysgetvalue{/pgfplots/table/@preprocessed cell content}\value
                \ifx\value\empty
                    \endgroup
                \else
                \pgfmathfloatparsenumber{\value}%
                \pgfmathfloattofixed{\pgfmathresult}%
                \let\value=\pgfmathresult
                %
                % map that value:
                \pgfplotscolormapaccess
                    [\pgfkeysvalueof{/color cells/min}:\pgfkeysvalueof{/color cells/max}]
                    {\value}
                    {\pgfkeysvalueof{/pgfplots/colormap name}}%
                % now, \pgfmathresult contains {<R>,<G>,<B>}
                % 
                % acquire the value AFTER any preprocessor or
                % typesetter (like number printer) worked on it:
                \pgfkeysgetvalue{/pgfplots/table/@cell content}\typesetvalue
                \pgfkeysgetvalue{/color cells/textcolor}\textcolorvalue
                %
                % tex-expansion control
                % see https://tex.stackexchange.com/questions/12668/where-do-i-start-latex-programming/27589#27589
                \toks0=\expandafter{\typesetvalue}%
                \xdef\temp{%
                    \noexpand\pgfkeysalso{%
                        @cell content={%
                            \noexpand\cellcolor[rgb]{\pgfmathresult}%
                            \noexpand\definecolor{mapped color}{rgb}{\pgfmathresult}%
                            \ifx\textcolorvalue\empty
                            \else
                                \noexpand\color{\textcolorvalue}%
                            \fi
                            \the\toks0 %
                        }%
                    }%
                }%
                \endgroup
                \temp
                \fi
            }%
        }%
    }
}


%*******************************************************
% Bibliography
%*******************************************************
%   Ficheiro com a base de dados da bibliografia
\addbibresource{References.bib}
%   para o kile dar as sugestões das chaves da bibliografia
%   se der erro a queixar-se do bibtex, basta repetir a compilação
\iffalse
    \bibliography{References.bib}  % só para o kile
\fi


%*******************************************************
% Lista de acrónimos
%*******************************************************
% \loadglsentries{Covers/Acronyms-list}
\makeglossaries


%*******************************************************
% Hyphenation
%*******************************************************
%\hyphenation{put special hyphenation here}


% ******************************************************
% GO!GO!GO! MOVE IT!
%*******************************************************
\begin{document}
\frenchspacing
\raggedbottom
\selectlanguage{portuguese}

\pagestyle{plain}

% use \cleardoublepage here to avoid problems with pdfbookmark

%*******************************************************
% Frontmatter
%*******************************************************
\include{Covers/Titlepage}
\cleardoublepage% Title Page

\begin{titlepage}

\begin{center}

\hfill

% \includegraphics[width=8cm]{Covers/ipl-estg} \\
\includegraphics[width=\textwidth]{Covers/estg_h.pdf} \\

\bigskip
\large
\myFaculty \\
\mySchool \\ 
\myDepartment \\
\myDegree \\



\begingroup
\color{Maroon}

\vspace{4cm}

\spacedallcaps{\myTitle} \\ \bigskip % Thesis title
\endgroup
% \mySubtitle \\ \medskip % Thesis subtitle

\vspace{4cm}

% \vfill


\spacedlowsmallcaps{\myNameOne}\\
Número: \myNumberOne \\
\spacedlowsmallcaps{\myNameTwo}\\
Número: \myNumberTwo \\
% \spacedlowsmallcaps{\myNameTwo}
\bigskip % Your name

\vfill
\begin{normalsize}
    
    \noindent Dissertação realizada sob orientação do \myProfOne, \myProfTwo e \myProfThree.

\end{normalsize}
\bigskip
\myLocation, \myTime\ %-- \myVersion % Time and version
\end{center}


\vspace{1cm}

\begin{center}


    
\end{center}



% \end{addmargin}

\end{titlepage}

\cleardoublepage



\pagenumbering{roman}
\cleardoublepage\include{Covers/Acknowledgments}
\cleardoublepage\include{Covers/Abstract}

\pagestyle{scrheadings}

\cleardoublepage\include{Covers/Contents}
\cleardoublepage\include{Covers/Acronyms}


%********************************************************************
% Mainmatter
%*******************************************************
\pagenumbering{arabic}

% \phantomsection 
% \part*{Relatório}


\cleardoublepage\include{Chapters/Intro}
\cleardoublepage\include{Chapters/Background}
\cleardoublepage\addtocontents{toc}{\protect\vspace{\beforebibskip}}% Place slightly below the rest of the document content in the table


%************************************************
\chapter{Desenvolvimento}
\label{ch:desenvolvimento}
%************************************************

\section{Teste 5}

\subsection{Objetivo}
Classificar Imagens com quadrados parciais e quadrados normais. É importante mencionar que os quadrados parciais estão todos com parte fora da imagem, nas extremidades. 

\subsection{Dataset}
O Dataset é composto por 11000 imagens de treino e 5000 de teste. Composto por 2 classes: 
\begin{itemize}
    \item Quadrados 
    \item Quadrados Parciais
\end{itemize}
\begin{figure}[H]
    \centering
    \includegraphics[width=0.35\linewidth]{imgs/Test_5/5/dataset/square_5501.png}
    \includegraphics[width=0.35\linewidth]{imgs/Test_5/5/dataset/square_cut_5501.png}
    \caption{Quadrado e Quadrado Parcial}
    \label{fig:enter-label}
\end{figure}
\begin{figure}[H]
    \centering
    \includegraphics[width=1.0\linewidth]{imgs/Test_5/5/dataset/Squares_Area_Distribution_Hist.png}
    \caption{Distribuição da Área (Quadrados)}
    \label{fig:enter-label}
\end{figure}
\begin{figure}[H]
    \centering
    \includegraphics[width=1.0\linewidth]{imgs/Test_5/5/dataset/Partial_Squares_Area_Distribution_Hist.png}
    \caption{Distribuição da Área Visivel (Quadrados Parcial)}
    \label{fig:enter-label}
\end{figure}

\subsection{Treino}

\begin{figure}[H]
    \centering
    \includegraphics[width=\textwidth]{imgs/Test_5/5/train_test_acc.png}
    \caption{Acurácia de Validação e de Treino}
    \includegraphics[width=\textwidth]{imgs/Test_5/5/train_test_loss.png}
    \caption{Perda de Validação e de Treino}
    \label{fig:sub2}
\end{figure}

Foram feitas 30 épocas, alcançando a melhor \texttt{val\_acc} na época 13 de \(94.70\%\).

\subsection{Amostras Mal Classificadas}

No total foram mal classificadas 265 (5.3\% ) imagens, sendo 95 (36\%) delas quadrados normais e as restantes 170 (64\%) quadrados parciais. 

\subsection{Matriz de Confusão}

\begin{table}[H]
\centering
\begin{tabular}{l|c c}
                   & Quadrados & Quadrados Parciais \\
\hline
Quadrados          & 2405         & 170                  \\
Quadrados Parciais & 95           & 2330                  \\
\end{tabular}
\end{table}

\subsection{Metricas de Avaliação}

\begin{table}[H]
\centering
\begin{tabular}{|c|c|c|}
\textit{Acuracy} & \textit{Precision} & \textit{Recall} \\
0.9470 & 0.9470 & 0.9470  \\
\end{tabular}
\end{table}

\subsection{Matriz de Correlação}

\subsection{Conclusão de Teste}
    Bla bla bla


\section{Teste 5.1}
\subsection{Objetivo}
Classificar imagens com vários quadrados parciais e imagens com vários quadrados normais. É importante mencionar que os quadrados parciais estão todos com parte fora da imagem, nas extremidades. 
\subsection{Dataset}
O Dataset é composto por 11000 imagens de treino e 5000 de teste. Composto por 2 classes: 
\begin{itemize}
    \item Quadrados (1 a 5 quadrados em cada imagem)
    \item Quadrados Parciais  (1 a 5 quadrados em cada imagem)
\end{itemize}

\begin{figure}[H]
    \centering
    \includegraphics[width=0.35\linewidth]{imgs/Test_5/5_1/dataset/square_5.png}
    \includegraphics[width=0.35\linewidth]{imgs/Test_5/5_1/dataset/square_cut_6.png}
    \caption{Quadrados e Quadrados Parciais}
    \label{fig:enter-label}
\end{figure}
\subsection{Treino}
\subsection{Amostras Mal Classificadas}
\subsection{Matriz de Confusão}
\subsection{Metricas de Avaliação}
\subsection{Matriz de Correlação}
\subsection{Conclusão de Teste}
    Bla bla bla


\section{Teste 6}
\subsection{Objetivo}
\subsection{Dataset}
\subsection{Treino}
\subsection{Amostras Mal Classificadas}
\subsection{Matriz de Confusão}
\subsection{Metricas de Avaliação}
\subsection{Matriz de Correlação}
\subsection{Conclusão de Teste}
    Bla bla bla


\section{Teste 6.1}
\subsection{Objetivo}
\subsection{Dataset}
\subsection{Treino}
\subsection{Amostras Mal Classificadas}
\subsection{Matriz de Confusão}
\subsection{Metricas de Avaliação}
\subsection{Matriz de Correlação}
\subsection{Conclusão de Teste}
    Bla bla bla

section{Teste 7}
\subsection{Objetivo}
    O Teste 7 consiste em descobrir quem está mais á direita se o quadrado ou um circulo,estes têm o tamanho igual.
\subsection{Dataset}
O Dataset é composto por 11000 imagens de treino e 5000 de teste. Composto por 2 classes:
\subsection{Treino}
\subsection{Amostras Mal Classificadas}
\subsection{Matriz de Confusão}
\subsection{Metricas de Avaliação}
\subsection{Matriz de Correlação}
\subsection{Conclusão de Teste}
    Bla bla bla

\newpage

\section{Teste 7.1}
\subsection{Objetivo}
    Ver quem está a direita
    Tamanhos iguais
    Com Intesecções
\subsection{Dataset}
O Dataset é composto por 11000 imagens de treino e 5000 de teste. Composto por 2 classes:
\begin{itemize}
        \item Circulo à direita
        \item Quadrado à direita
    \end{itemize}
    Cada imagem tem 2 formas, contendo uma das seguintes combinações:
    \begin{itemize}
        \item Circulo com Circulo
        \item Quadrado com Quadrado
        \item Circulo com Quadrados
    \end{itemize}
    \begin{figure}[H]
        \centering
            \includegraphics[width=0.25\linewidth]{imgs//Test_7/7_1/dataset/car_5629.png}
            \includegraphics[width=0.25\linewidth]{imgs//Test_7/7_1/dataset/car_7413.png}
        \caption{Circulo à direita}
        \label{fig:enter-label}
    \end{figure}
    \begin{figure}[H]
        \centering
        \includegraphics[width=0.25\linewidth]{imgs//Test_7/7_1/dataset/sar_5705.png}
        \includegraphics[width=0.25\linewidth]{imgs//Test_7/7_1/dataset/sar_7153.png}
        \caption{Quadrado à direita}
        \label{fig:enter-label}
    \end{figure}
    \begin{figure}[H]
        \centering
        \includegraphics[width=1.0\linewidth]{imgs//Test_7/7_1/dataset/distribution_squares.png}
        \caption{Distribuição da Área (Circulos à direita)}
        \label{fig:enter-label}
    \end{figure}
    \begin{figure}[H]
        \centering
        \includegraphics[width=1.0\linewidth]{imgs//Test_7/7_1/dataset/distribution_circles.png}
        \caption{Distribuição da Área (Quadrados à direita)}
        \label{fig:enter-label}
    \end{figure}
\subsection{Treino}
\begin{figure}[H]
    \centering
    \includegraphics[width=\textwidth]{imgs//Test_7/7_1/train_test_acc.png}
    \caption{Acurácia de Validação e de Treino}
    \includegraphics[width=\textwidth]{imgs//Test_7/7_1/train_test_loss.png}
    \caption{Perda de Validação e de Treino}
    \label{fig:sub2}
    Com as 26 épocas realizadas, conseguimos uma val acc de 0.9024, sendo que a melhor loss foi atingida na época 10.
\end{figure}
\subsection{Amostras Mal Classificadas}
No total foram mal classificadas 339 (13.74\% ) imagens de teste sendo:
    - 65 imagens com circulo à direita com um quadrado à esquerda
    - 44 imagens com circulo à direita e à esquerda
    - 86 imagens com quadrado à direita e com um circulo à esquerda 
    - 144 imagens com quadrado à direita e à esquerda
    

\subsection{Matriz de Confusão}

\begin{table}[H]
\centering
\begin{tabular}{l|c c}
                & Circulo à Direita & Quadrado à Direita \\ 
\hline
Circulo à Direita         & 2391         & 109                  \\
Quadrado à Direita        & 230         & 2270                 \\

\end{tabular}
\end{table}

\subsection{Metricas de Avaliação}
\subsection{Matriz de Correlação}
\subsection{Conclusão de Teste}
    Bla bla bla

\newpage

\section{Teste 7.2}
\subsection{Objetivo}
    Este teste é bastante semelhante ao teste 7, ao seja o objetivo é identificar entre um quadrado ou circulo, qual o que está mais á direita na imagem. Neste caso os tamanhos das formas podem ser diferente um do outro
\subsection{Dataset}
O Dataset é composto por 11000 imagens de treino e 5000 de teste. Composto por 2 classes:
    \begin{itemize}
        \item Circulo à direita
        \item Quadrado à direita
    \end{itemize}
    Cada imagem tem 2 formas, contendo uma das seguintes combinações:
    \begin{itemize}
        \item Circulo com Circulo
        \item Quadrado com Quadrado
        \item Circulo com Quadrados
    \end{itemize}
    \begin{figure}[H]
        \centering
            \includegraphics[width=0.25\linewidth]{imgs/Test_7/7_2/dataset/car_1.png}
            \includegraphics[width=0.25\linewidth]{imgs/Test_7/7_2/dataset/car_3116.png}
        \caption{Circulo à direita}
        \label{fig:enter-label}
    \end{figure}
    \begin{figure}[H]
        \centering
        \includegraphics[width=0.25\linewidth]{imgs/Test_7/7_2/dataset/sar_9.png}
        \includegraphics[width=0.25\linewidth]{imgs/Test_7/7_2/dataset/sar_3873.png}
        \caption{Quadrado à direita}
        \label{fig:enter-label}
    \end{figure}
    \begin{figure}[H]
        \centering
        \includegraphics[width=1.0\linewidth]{imgs/Test_7/7_2/dataset/car_ci_distribution_hist.png}
        \caption{Distribuição da Área (Circulos à direita)}
        \label{fig:enter-label}
    \end{figure}
    \begin{figure}[H]
        \centering
        \includegraphics[width=1.0\linewidth]{imgs/Test_7/7_2/dataset/sar_sq_distribution_hist.png}
        \caption{Distribuição da Área (Quadrados à direita)}
        \label{fig:enter-label}
    \end{figure}
\subsection{Treino}
\begin{figure}[H]
    \centering
    \includegraphics[width=\textwidth]{imgs/Test_7/7_2/train_test_acc.png}
    \caption{Acurácia de Validação e de Treino}
    \includegraphics[width=\textwidth]{imgs/Test_7/7_2/train_test_loss.png}
    \caption{Perda de Validação e de Treino}
    \label{fig:sub2}
    Com as 20 épocas realizadas, conseguimos uma val acc de 0.9720, sendo que a melhor loss foi atingida na época 10.
\end{figure}

\subsection{Amostras Mal Classificadas}

No total foram mal classificadas 140 (2.8\% ) imagens sendo:
 \begin{itemize}
    \item 0 imagens com circulo à direita com um quadrado à esquerda
    \item 13 imagens com circulo à direita e à esquerda
    \item 45 imagens com quadrado à direita e com um circulo à esquerda 
    \item 52 imagens com quadrado à direita e à esquerda
\end{itemize}
    

\subsection{Matriz de Confusão}

\begin{table}[H]
\centering
\begin{tabular}{l|c c}
                & Circulo à Direita & Quadrado à Direita \\
\hline
Circulo à Direita         & 2406         & 97                  \\
Quadrado à Direita        & 43         & 2457                 \\

\end{tabular}
\end{table}

\subsection{Análise}
\begin{figure}[H]
    \centering
    \includegraphics[width=\textwidth]{imgs/Test_7/7_2/failed/car_ci_failed_area_hist.png}
    \caption{Distribuição da Área dos Circulos à direita }
    \label{fig:sub2}
\end{figure}

\begin{figure}[H]
    \centering
    \includegraphics[width=\textwidth]{imgs/Test_7/7_2/failed/sar_sq_failed_area_hist.png}
    \caption{Distribuição da Área dos Quadrados à direita }
    \label{fig:sub2}
\end{figure}

\begin{figure}[H]
    \centering
    \includegraphics[width=\textwidth]{imgs/Test_7/7_2/failed/car_ci_failed_right_left_scatter.png}
    \caption{Scatter dos circulos, em imagens de circulos à direita}
    \label{fig:sub2}
\end{figure}

\begin{figure}[H]
    \centering
    \includegraphics[width=\textwidth]{imgs/Test_7/7_2/failed/car_ci_failed_right_left_scatter.png}
    \caption{Scatter dos quadrados, em imagens de quadrados à direita}
    \label{fig:sub2}
\end{figure}

Como podemos ver partir dos histogramas, o que causa o modelo ao engano é quando numa imagem está presente circulos ou quadrados muito pequenos. Um pormenor que é possivel observar, principalmente no scatter dos quadrados, é que o que está á esquerda não influencia muito devido ao tamanho, o tamanho só influencia caso esse esteja à direita


\section{Teste 7.3}
\subsection{Objetivo}
    Ver quem está a direita
    Tamanhos diferentes
    Com Intersecções

 	    
 

\section{Teste 9}
\subsection{Objetivo}
    Classificar Imagens com quadrados, circulos e vazios. Ao seja, um problema de classificação não binário.
\subsection{Dataset}
    O Dataset é composto por 10998 imagens de treino e 4998 de teste. Composto por 3 classes: 
    \begin{itemize}
        \item Circulos 
        \item Quadrados
        \item Vazios
    \end{itemize}
    \begin{figure}[H]
        \centering
        \includegraphics[width=0.25\linewidth]{imgs/Test_9/dataset_9/circles_4.png}
        \includegraphics[width=0.25\linewidth]{imgs/Test_9/dataset_9/squares_4.png}
        \includegraphics[width=0.25\linewidth]{imgs/Test_9/dataset_9/nones_3.png}
        \caption{Circlos, Quadrados e Vazios}
        \label{fig:enter-label}
    \end{figure}
    \begin{figure}[H]
        \centering
        \includegraphics[width=1.0\linewidth]{imgs/Test_9/dataset_9/Squares_Area_Distribution_Hist.png}
        \caption{Distribuição da Área (Quadrados)}
        \label{fig:enter-label}
    \end{figure}
    \begin{figure}[H]
        \centering
        \includegraphics[width=1.0\linewidth]{imgs/Test_9/dataset_9/Circles_Area_Distribution_Hist.png}
        \caption{Distribuição da Área (Circulos)}
        \label{fig:enter-label}
    \end{figure}

\subsection{Treino}

\begin{figure}[H]
    \centering
    \includegraphics[width=\textwidth]{imgs/Test_9/train_test_acc.png}
    \caption{Acurácia de Validação e de Treino}
    \includegraphics[width=\textwidth]{imgs/Test_9/train_test_loss.png}
    \caption{Perda de Validação e de Treino}
    \label{fig:sub2}
    Com as 23 épocas realizadas, conseguimos uma val acc de 0.9876, sendo que a melhor loss foi atingida na época 13.
\end{figure}

\subsection{Amostras Mal Classificadas}

No total foram mal classificadas 66 (1.32\% ) imagens, sendo 26 (39.39\%) delas circulos, 18 (27.27\%) quadrados e as restantes 22 (33.3\%) vazios. 

\subsection{Matriz de Confusão}

\begin{table}[H]
\centering
\begin{tabular}{l|c c c}
                 & Circulo & Quadrado & Vazio \\
\hline
Circulo          & 0.98       & 0.02       & 0.000059          \\
Quadrado    & 0.02       & 0.98       & 0.000058           \\
Vazio             & 0            & 0            & 1           \\
\end{tabular}
\end{table}

\subsection{Análise}

\newpage

\section{Teste 10}
\subsection{Objetivo}
    Classificar Imagens com quadrados, circulos e triangulos. Ao seja, um problema de classificação não binário.
\subsection{Dataset}
    O Dataset é composto por 10998 imagens de treino e 4998 de teste. Composto por 3 classes: 
    \begin{itemize}
        \item Circulos 
        \item Quadrados
        \item Triangulos
    \end{itemize}
    \begin{figure}[H]
        \centering
        \includegraphics[width=0.25\linewidth]{imgs/Test_10/dataset_10/circles_9.png}
        \includegraphics[width=0.25\linewidth]{imgs/Test_10/dataset_10/squares_9.png}
        \includegraphics[width=0.25\linewidth]{imgs/Test_10/dataset_10/triangles_9.png}
        \caption{Circlos, Quadrados e Triangulos}
        \label{fig:enter-label}
    \end{figure}
    \begin{figure}[H]
        \centering
        \includegraphics[width=1.0\linewidth]{imgs/Test_9/dataset_9/Squares_Area_Distribution_Hist.png}
        \caption{Distribuição da Área (Quadrados)}
        \label{fig:enter-label}
    \end{figure}
    \begin{figure}[H]
        \centering
        \includegraphics[width=1.0\linewidth]{imgs/Test_9/dataset_9/Circles_Area_Distribution_Hist.png}
        \caption{Distribuição da Área (Circulos)}
        \label{fig:enter-label}
    \end{figure}

\subsection{Treino}

\begin{figure}[H]
    \centering
    \includegraphics[width=\textwidth]{imgs/Test_9/train_test_acc.png}
    \caption{Acurácia de Validação e de Treino}
    \includegraphics[width=\textwidth]{imgs/Test_9/train_test_loss.png}
    \caption{Perda de Validação e de Treino}
    \label{fig:sub2}
    Com as 23 épocas realizadas, conseguimos uma val acc de 0.9876, sendo que a melhor loss foi atingida na época 13.
\end{figure}

\subsection{Amostras Mal Classificadas}

No total foram mal classificadas 66 (1.32\% ) imagens, sendo 26 (39.39\%) delas circulos, 18 (27.27\%) quadrados e as restantes 22 (33.3\%) vazios. 

\subsection{Matriz de Confusão}

\begin{table}[H]
\centering
\begin{tabular}{l|c c c}
                 & Circulo & Quadrado & Vazio \\
\hline
Circulo          & 2405         & 22     & 1           \\
Quadrado         & 95           & 2330   & 1           \\
Vazio            & 95           & 2330   & 1           \\
\end{tabular}
\end{table}


\section{Teste 8 R1}
\subsection{Objetivo}

\subsection{Dataset}
O Dataset é composto por 5500 imagens de treino e 2500 de teste. Composto por apenas imagens com 1 quadrado. 
Este dataset foi tambem utilizado no teste 5.
    \begin{figure}[H]
        \centering
        \includegraphics[width=0.25\linewidth]{imgs/Test_8/dataset/square_22.png}
        \label{fig:enter-label}
    \end{figure}
\subsection{Treino}
	\begin{figure}[H]
	    \centering
	      \includegraphics[width=1.0\linewidth]{imgs/Test_8/loss.png}
	    \caption{Loss}
	        \includegraphics[width=1.0\linewidth]{imgs/Test_8/mae.png}
	    \caption{MAE}
	    \label{fig:sub2}
	\end{figure}
	  Com as 50 épocas realizadas, conseguimos obter previsões das áreas com uma margem de erro por volta de 2000px de área.
	
 \subsection{Análise}       
        \begin{figure}[H]
	        \centering
	        \includegraphics[width=1.0\linewidth]{imgs/Test_8/area_predict_true_line.png}
	          \caption{área prevista e área verdadeira}
	        \label{fig:enter-label}
        \end{figure}
        \begin{figure}[H]
	        \centering
	        \includegraphics[width=1.0\linewidth]{imgs/Test_8/area_diference_hist.png}
	        \caption{Diferença entre area prevista e a area verdadeira}
	        \label{fig:enter-label}
    \end{figure}
    
   \section{Teste R2}
\subsection{Objetivo}

\subsection{Dataset}
O Dataset é composto por 5500 imagens de treino e 2500 de teste. Composto por apenas imagens com 1 quadrado. 
Este dataset foi tambem utilizado no teste 5.
    \begin{figure}[H]
        \centering
        \includegraphics[width=0.25\linewidth]{imgs/Test_8/dataset/square_22.png}
        \label{fig:enter-label}
    \end{figure}
\subsection{Treino}
	\begin{figure}[H]
	    \centering
	      \includegraphics[width=1.0\linewidth]{imgs/Test_8/loss.png}
	    \caption{Loss}
	        \includegraphics[width=1.0\linewidth]{imgs/Test_8/mae.png}
	    \caption{MAE}
	    \label{fig:sub2}
	\end{figure}
	  Com as 50 épocas realizadas, conseguimos obter previsões das áreas com uma margem de erro por volta de 2000px de área.
	
 \subsection{Análise}       
        \begin{figure}[H]
	        \centering
	        \includegraphics[width=1.0\linewidth]{imgs/Test_8/area_predict_true_line.png}
	          \caption{área prevista e área verdadeira}
	        \label{fig:enter-label}
        \end{figure}
        \begin{figure}[H]
	        \centering
	        \includegraphics[width=1.0\linewidth]{imgs/Test_8/area_diference_hist.png}
	        \caption{Diferença entre area prevista e a area verdadeira}
	        \label{fig:enter-label}
    \end{figure}
    
\cleardoublepage\include{Chapters/Conclusion} 

\cleardoublepage\include{Covers/Bibliography}


%********************************************************************
% Backmatter
%*******************************************************
\appendix

\cleardoublepage
\phantomsection 
\part*{Apêndices}

\cleardoublepage\include{Chapters/ApendixA}

\cleardoublepage\include{Chapters/ApendixB}

\cleardoublepage\include{Covers/Declaration}
%********************************************************************
% Game Over: Restore, Restart, or Quit?
%*******************************************************
\end{document}
%********************************************************************
